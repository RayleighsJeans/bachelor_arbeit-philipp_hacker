% Copyright (C)  2015  Philipp Hacker.
% Permission is granted to copy, distribute and/or modify this document
% under the terms of the GNU Free Documentation License, Version 1.3
% or any later version published by the Free Software Foundation;
% with no Invariant Sections, no Front-Cover Texts, and no Back-Cover Texts.
% The lincense itself can be found at <https://www.gnu.org/licenses/fdl-1.3>.

\documentclass[numbers=noenddot,12pt,a4paper,bibtotoc]{scrartcl}

\usepackage[greek,ngerman]{babel}
\usepackage[T1]{fontenc}
\usepackage[utf8]{inputenc}
\usepackage{fullpage}
\usepackage{libertine}
\usepackage{ziffer}
\usepackage{graphicx}
\usepackage{units}
\usepackage[infoshow]{tabularx}
\usepackage{amsmath}
\usepackage{amssymb}
\usepackage{wrapfig}
\usepackage{upgreek}
\usepackage{esint}
\usepackage{float}
\usepackage{wrapfig}
\usepackage[font=small,labelfont=bf]{caption}
\usepackage{subcaption}
\usepackage{lscape}
\usepackage{hyperref}
\usepackage{csquotes}

\renewcommand{\thefigure}{Abb. \arabic{figure}}

\captionsetup[wrapfigure]{name=}
\captionsetup[figure]{name=}

\newcommand{\nummat}[1]{\left[\text{#1}\right]}
\newcommand{\num}[1]{$\left[\text{#1}\right]$}
\newcommand{\degree}{^\circ}
\newcommand{\diff}{\textnormal{d}}
\newcommand{\tenpo}[1]{\cdot 10^{#1}}
\newcommand{\greek}[1]{\greektext#1\latintext}
\newcommand{\ix}[1]{_\text{#1}}
\newcommand{\imag}{\mathbf{i}}
\newcommand{\tilt}[1]{\textit{#1}}
\newcommand{\grad}[1]{\textit{grad}\left(#1\right)}
\newcommand{\divergenz}[1]{\textit{div}\left(#1\right)}
\newcommand{\euler}{\mathnormal{e}}
\newcommand{\fett}[1]{\textbf{#1}}

\title{Bachelor-Arbeit zum Thema \enquote{Modenanregung in \tilt{Yukawa}-Bällen}} %TODO Name
\author{Philipp Hacker} %TODO Author
\date{\today}

\setcounter{section}{-1}


\begin{document}
	
	\maketitle
	\centering 
	Institut für Physik\\
	mathematisch-naturwissenschaftliche Fakultät\\
	Universität Greifswald
	
	\vspace{0.5cm}
	
	\begin{figure}[H]
			\centering
			\includegraphics[width=0.35\textwidth]{figs/unilogo_NEU_schwarz.eps}
	\end{figure}
	
	\vspace{0.5cm}
	
	\begin{center}
			\hspace{-0.55cm} Erst-Gutachter: Prof. Dr. André Melzer \\ \vspace{0.25cm} %TODO Name Erst-Gutachter
			
			Zweit-Gutachter: Prof. Dr. Lutz Schweikhard \\ \vspace{0.25cm} %TODO Name Zweit-Gutachter
			
			Bearbeitungszeitraum: 01.03.2015 bis 12.07.2015 \\ \vspace{0.25cm} %TODO Bearbeitungszeitraum
		
%		\begin{table}[h]
%			\centering
%			Note (Erst-Gutachter): %TODO Gute Note erhalten :)
%			\begin{tabularx}{1.5cm}{|X|}
%				\hline \\ \\
%				\hline
%			\end{tabularx}
%			
%			\centering
%			\hspace{-0.42cm} Note (Zweit-Gutachter): %TODO Gute Note erhalten :)
%			\begin{tabularx}{1.5cm}{|X|}
%				\hline \\ \\
%				\hline
%			\end{tabularx}
%			
%		\end{table}

	\end{center}
	
	\thispagestyle{empty}
	
	\newpage
	
	\tableofcontents
	
	\newpage
	
	\section{Motivation}\label{sec:einleitung}
	
		
	\newpage
	
	\section{Physikalische Grundlagen}\label{sec:physg}
	
		
	\newpage
	
	\section{Durchführung}\label{sec:durch}
	
		
	\newpage
	
	\section{Auswertung}\label{sec:auswert}
	
\end{document}