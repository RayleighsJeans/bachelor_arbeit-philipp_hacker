\chapter{Physikalische Grundlagen}\label{sec:physg}

  In diesem ersten Abschnitt soll der Hintergrund f\"ur den sich anschlie{\ss}enden Versuch gegeben werden. Dabei wird einerseits auf Grundlegendes aus der Plasmaphysik sowie komplexer Plasmen eingegangen, andererseits aber auch auf die theoretischen Modelle, welche essentiell f\"ur das Verst\"andnis der Ph\"anomene dieses Experiments sind.

  \section{Kapazitiv gekoppelte Radiofrequenz-Plasmen} \label{sub:kaprfplasm}

    Der in diesem Versuch genutzte Aufbau entspricht dem eines kapazitiv gekoppelten Niederdruck-Radiofreuquenz-Plasmas.\\
    Ein Plasma ist ein quasineutrales Gas aus freien Ladungsträgern und, dem Ionisierungsgrad der Entladung entsprechend, neutralen Atomen oder Molekülen. Die Spezies der Ladungen sind im Allgemeinen Elektronen und Ionen, wobei der Begriff quasineutral die Bedingung $n\ix{e}=n\ix{I}$ der Dichten auf einer speziellen Längenskala fordert (siehe unten). Für ein Plasma gibt es verschiedene physikalische Kenngrößen, welche in Tabelle \ref{tab:kenngroessen} inklusive ihrer Bedeutungen einmal zusammengefasst wurden.\\
    Befindet sich ein Fremdteilchen, ein Festkörper oder eine weitere Ladungsträgerspezies in der Entladung, so spricht man von komplexen Plasmen. Für das Experiment dieser Arbeit ist dies der Fall, da hierbei in das Plasma homogene Melamin-Formaldehyd-Partikel in der Größenordnung einiger $\unit{\upmu m}$ eingebracht werden. Diese erfahren, in Abhängigkeit der Parameter der Gasentladung, verschiedenste Wechselwirkungen mit den Ladungsträgern und externen Feldern.

      \begin{table}[H]
        \centering
          \begin{tabular}{m{0.3\textwidth}|m{0.3\textwidth}|m{0.3\textwidth}}
            Größe & Zusammenhang & Bedeutung \\ 
            \hline  Debye-Länge & $\lambda\ix{D,j}^2=\frac{\varepsilon\ix{0}k\ix{B}T\ix{j}}{n\ix{j}e^2}$
            \newline
            $\lambda\ix{D}^2=\left(\lambda\ix{D,e}^{-2}+\lambda\ix{D,I}^{-2}\right)^{-1}$ & die Distanz um eine Probeladung, ab welcher die Quasineutralität gilt\\ 

            \hline Plasmafrequenz & $\omega\ix{P,j}^2=\frac{n\ix{j}e^2}{\varepsilon\ix{0}m\ix{j}}=\frac{v\ix{th,j}}{\lambda\ix{D,j}}=\frac{1}{\tau\ix{j}}$ & obere Grenze der Zeitskala für die Wechselwirkung mit externen Kräften bzw. Feldern; Inverse der Abschirmungszeit \\ 

            \hline thermisches Geschwindigkeit & $v\ix{th,j}^2=\frac{k\ix{B}T\ix{j}}{m\ix{j}}$ & mittlere Geschwindigkeit aus der Definition der kinetischen Gastheorie \\ 

            \hline mittlerer Teilchenabstand & $\overline{b}=\frac{\hbar}{m\ix{j}v\ix{th,j}}$ & gemittelter Teilchenabstand innerhalb der Spezies $j$ \\ 

            \hline Yukawa-Potential & $\Phi=\frac{Q}{4\pi\varepsilon|\vec{r}|}\euler^{-\frac{|\vec{r}|}{\lambda\ix{D}}}$ & elektrostatisches Wechselwirkungspotential einer Probeladung $Q$ in einem Plasma \\

            \hline

          \end{tabular}
        \caption{Plasmaphysikalische Kenngrößen. Der Index $j$ bestimmt die Ladungsträgerspezies. Dabei sind $m\ix{j}$ Massen, $n\ix{j}$ Dichten, $v\ix{th,j}$ thermische Geschwindigkeiten und $T\ix{j}$ Temperaturen. Die Indizees $I$ und $e$ kennzeichnen im Folgenden die Größen der Ionen und Elektronen}
        \label{tab:kenngroessen}
      \end{table}

    Wie bereits erwähnt, greift dieses Experiment auf die Erzeugung einer Gasentladung zurück. Hierbei sind die Elektroden wie ein horizontal ausgerichteter, paralleler Plattenkondensator angeordnet (siehe \autoref{img:ungleichesplasma}), wobei das "`Dielektrikum"' das Plasma sei, die untere Elektrode mit einem Signal im $\unit{MHz}$-Bereich betrieben wird und die andere auf Massepotential liegt. Das elektrische Feld resultiert hierbei aus den Oberflächenladungen der Elektroden. Mit unterschiedlichen \tilt{rf}-Signalen können (\tilt{radio frequency}) auch verschiedene Betriebsregime realisiert werden: die Erzeugung und Vernichtung der Ladungsträger verändert sich.\\
    Die \tilt{rf}-Spannung sorgt weiterhin für einen Verschiebungsstrom zwischen Entladungsvolumen und Elektrode (siehe Abschnitt \ref{subsub:self-bias}). In Aufbauten, in welchen die Flächen der Elektroden nicht gleich sind, liegt außerdem wegen der unterschiedlichen Potentialverläufe in den Randschichten (siehe Abschnitt \ref{sub:rand}) zusätzlich eine Gleichspannung an, die sogenannte \tilt{self-bias} - siehe Abschnitt \ref{subsub:self-bias}. Das ist insbesondere der Fall für das Experiment dieser Arbeit, da hier die gesamte Kammer als Elektrode zur Masse dient und die andere entsprechend kapazitiv an die Erregung gekoppelt ist.

      \subsection{Verschiebungsstrom} \label{subsub:verschieb}

        \begin{wrapfigure}{l}{0.45\textwidth}
          \centering
          \includegraphics[width=0.42\textwidth,height=0.5\textwidth]{figs/verschiebpiel.png}
          \caption{Verlauf der Dichten, des Elektrischen Feldes und des zug. Potentials (eindimensional) in einer Entladung mit zwei gegenüberliegenden Randschichten (nach \cite{Piel10}).}
          \label{img:verschieb}
        \end{wrapfigure}

      Die Elektronen eines Plasmas über einer, mit Radiofrequenz getriebenen Elektrode können, aufgrund ihrer Beweglichkeit und hohen Plasmafrequenz $\omega\ix{P,e}$, der Anregung problemlos folgen. Die Ionen sollen in diesem Fall als stationär betrachtet werden, da $\omega\ix{P,I}\ll\omega\ix{P,e}$ und ganz besonders $\omega\ix{p,I}\ll\omega\ix{rf}$ der Frequenz der Entladung gilt. Nimmt man an, dass die Randschicht der Dicke $a$ vor einer negativ geladenen Wand frei von Elektronen ist (siehe Abschnitt \ref{sub:rand}) und die Ionendichte dort konstant $n\ix{0,I}$ beträgt, so folgt für das elektrische Feld der positiven Raumladungszone $E\ix{0}=-en\ix{0,I}a/\varepsilon\ix{0}\,\,$ - siehe \autoref{img:verschieb}. Lädt sich die Kammerwand nun stärker negativ auf, so dehnt sich die Randschicht als Folge dessen aus und wandert dabei mit der Geschwindigkeit $v=\diff s\ix{1}/\diff t$ in das Plasma-Volumen hinein. Dies führt zum bereits erwähnten Verschiebungstrom in der Randschicht $j\ix{V}=-en\ix{0,I}v$, welcher zusammen mit dem Strom der, aus der Schicht "`verdrängten"' Elektronen $j\ix{L}=-en\ix{0,I}v$ die Kontinuität im Plasma erhält (nach \cite{Godyak90d}). Da im Hauptvolumen der Entladung die Quasineutralität erhalten ist, dh. $n\ix{e}=n\ix{0,I}$ gilt, müssen die zusätzlichen Elektronen aus der pos. Ladungszone in $x\in\left[0,s\ix{1}+\diff s\ix{1}\right]$ in den Teil des Plasmas, in welchem zu diesem Zeitpunkt die Randschicht schrumpft und $\diff s\ix{1}=-\diff s\ix{2}$ gilt. Daraus folgt wiederum, dass die Randschichten eines harmonisch getriebenen Plasmas sinusoidal um eine Gleichgewichtsdicke $s\ix{0}$ schwingen und während einer jeden Periode einmal vollständig verschwinden, e.g. $s\ix{1/2,max}=s\ix{0}\,$.  Für den Spannungsabfall über das Plasma zwischen den Randschichten gilt nach \cite{Piel10} \autoref{eq:spannungsdiff}, wobei $U\ix{1/2}$ die Potentiale in den Randbereichen sind. 

        \begin{align}
          \Delta U=U\ix{1}-U\ix{2}=-\frac{2en\ix{0,I}s\ix{0}}{\varepsilon\ix{0}}\exp\left(\imag\omega t\right)\,\,. \label{eq:spannungsdiff}
        \end{align}

      \subsection{\tilt{self-bias}}  \label{subsub:self-bias}

          \begin{figure}[!t]
            \centering
            \begin{subfigure}[b]{0.48\textwidth}
              \centering
              \includegraphics[width=0.7\textwidth,height=0.8\textwidth]{figs/schaltbildselfbiaspiel1.png}
              \caption{}
              \label{img:ungleichesplasma}
            \end{subfigure}
            \begin{subfigure}[b]{0.48\textwidth}
              \centering
              \includegraphics[width=0.4\textwidth,height=0.8\textwidth]{figs/schaltbildselfbiaspiel2.png}
              \caption{}
              \label{img:ersatzschaltbild}
            \end{subfigure}
            \caption{\fett{(a)}:Schema eines Plasmas mit ungleichen Elektrodenflächen. Die obere ist hierbei kleiner und gleichzeitig mit dem \tilt{rf}-Signal betrieben. \fett{(b)}:Ersatzschaltbild einer Entladung mit der \tilt{bulk}-Impedanz $Z\ix{P}$ (Hauptvolumen) und den Randschichten: eine Diode als Symbol für den "`Elektronenfluss"' aus der Schicht hinaus, der Widerstand $R\ix{j}$ als Ionenstrom und die Kapazität $C\ix{j}$ der pos. Raumladungszone (nach \cite{Piel10}).}
          \end{figure}

        Einem Plasma kann man eine, für \tilt{rf}-Spannungen der Frequenz $\omega$ nicht verschwindende Impedanz $Z\ix{P}$ zuordnen. Für das Entladungs-Volumen der Permittivität $\varepsilon\ix{P}$, der Kapazität $C\ix{P}$ eines Quaders der Querschnittsfläche $A$ und Dicke $b$, in welcher die freien Elektronen mit der Frequenz $\nu\ix{e,N}$ mit den Neutralgasatomen stoßen, gilt (\cite{Piel10}):

        \begin{align}
        \varepsilon\ix{P}=1-&\frac{\omega\ix{P,e}^2}{\omega\left(\omega-\imag\nu\ix{e,N}\right)} \quad \quad C\ix{P}=\varepsilon\ix{P}C\ix{0}=\varepsilon\ix{P}\varepsilon\ix{0}\frac{A}{b} \\
        &Z\ix{P}=\left(\imag\omega C\ix{P}+ \frac{1}{\frac{1}{\omega\ix{P,e}^2C\ix{0}}\left(\nu\ix{e,N}+\imag\omega\right)}\right)^{-1}
        \label{eq:bulkimpedanz}
        \end{align}

        Die Induktivität $\imag\omega/\left(\omega\ix{P,e}^2C\ix{0}\right)$ des "`Schaltkreises der Entladung"' gibt die Trägheit der Elektronen in Bezug auf ein Signal $\omega$ wieder, wohingegen der reele Widerstand $\nu\ix{e,N}/\left(\omega\ix{P,e}^2C\ix{0}\right)$ der Neutralgasreibung entspricht. Zusammen mit den Randschichten ergibt sich das Ersatzschaltbild eines Plasmas in \autoref{img:ersatzschaltbild}.\\
        Bei hohen Anregungsfrequenzen, wie es in diesem Experiment der Fall ist, kann man die Impedanz des \tilt{bulk} (siehe \autoref{eq:bulkimpedanz}, \cite{Kay85}) vernachlässigen und es dominieren die Kapazitäten $C\ix{j}$ der Randschichten beider Elektroden. Es folgt für die Spannung über die Entladung $U$ und das Plasmapotential $\Phi\ix{P}$:

          \begin{align}
            U\left(t\right)=U\ix{GS}+U\ix{rf}\sin\left(\omega t\right) \quad \quad \Phi\ix{P}\left(t\right)=\overline{\Phi\ix{P}}+\Phi\ix{rf}\sin\left(\omega t\right) \label{eq:selfbiaseins}
          \end{align}

        Wie bereits erwähnt, verschwindet die Randschicht vollständig während einer Anregungsperiode. Als Folge dessen stellt sich lokal $\Phi\ix{P}$ auf das Potential der Elektrode ein, woraus wiederum ein Elektronenstrom auf diese folgt. Es kommt sozusagen zu einem Kurzschluss in der Randschicht, wenn das Plasma-Potential negativ im Vergleich zur Elektrode wird. Mit \autoref{eq:selfbiaseins}) zusammen (nach \cite{Piel10}) ergibt sich \autoref{eq:ungleichungen}.

          \begin{align}
            \Phi\ix{P,max}=\overline{\Phi\ix{P}}+\Phi\ix{rf}\geq U\ix{Gs}+U\ix{rf} \quad \quad \Phi\ix{P,min}=\overline{\Phi\ix{P}}-\Phi\ix{rf}\geq 0\, \, . \label{eq:ungleichungen}
          \end{align}

          \begin{wrapfigure}{r}{0.43\textwidth}
            \centering
            \vspace{-0.5cm}
            \includegraphics[width=0.35\textwidth,height=0.35\textwidth]{figs/kapazitivekopplungohneschemapiel.png}
            \caption{ Schema des Potential- und Spannungsverlaufs einer direkt und kapazitiv gekoppelten \tilt{rf}-Elektrode (nach \cite{Piel10}).}
            \label{img:kapazitivgekoppelt}
          \end{wrapfigure}

        Liegt die getriebene Elektrode ohne zwischengeschalteten ``Puffer'' direkt an der \tilt{rf}-Signalquelle, so gilt zumindest f\"ur eine Ungleichung aus (\ref{eq:ungleichungen}) die Gleichheit. Wird hingegen die Elektrode kapazitiv gekoppelt, dh. zwischen diese und den Generator ein Kondensator geschaltet, so kann in einer Periode der \tilt{rf}-Anregung kein Nettostrom von der Quelle flie{\ss}en. Der Strom auf die Elektrode kann während eines \tilt{rf}-Zyklus die Kapazität nicht umladen. Deswegen gibt es gleiche Elektronenstr\"ome auf beide Elektroden, was zur Folge hat, dass das minimale Plasmapotential das Massepotential wird und das maximale zu dem der Anregung. F\"ur den Gleichspannungsanteil, \tilt{self-bias} $U\ix{GS}$ und das mittlere Plasmapotential $\overline{\Phi\ix{P}}$ folgt:

          \begin{align}
            \overline{\Phi\ix{P}}=\frac{1}{2}\left(U\ix{GS}+U\ix{rf}\right) \quad \quad U\ix{GS}=\frac{C\ix{1}-C\ix{2}}{C\ix{1}+C\ix{2}}U\ix{rf} \,\, .\label{eq:selfbiaszwei} 
          \end{align}

%					Ist die Frequenz klein respektive \"ublicher Zeitskalen der Entladung, wie bspw. die Elektronen- oder Ionen-Plasmafrequenz (siehe \ref{tab:kenngroessen}), so wird der Strom aus den verdr\"angten Elektronen $j\ix{L}$ gr\"o{\ss}er als der Verschiebungsstrom durch die Randschichtschwankung $j\ix{V}\,$. Demzufolge wird der Strom aus Elektronen auf die getriebenen Elektrode, im Vergleich zu dem aus Ionen, durch einen Maxwell-Faktor in Abhängigkeit der angelegten Spannung verkleinert. Die Impedanz der Randschicht über der Elektrode ist deswegen wesentlich größer als die der anderen bzw. der Wände. Mit den \autoref{eq:selfbiaseins} und \autoref{eq:ungleichungen} folgt, dass das Plasma-Potential näherungsweise verschwindet und deswegen nur die Kontinuität für den Strom auf die getriebene Elektrode erhalten sein muss. Der \tilt{self-bias} bei kleinen Anregungsfrequenzen ergibt sich in \autoref{eq:selfbiasdrei} ($\mathbf{J}\ix{0}$ \tilt{Bessel-Funktion}, nach \cite{Piel10}). 
%
%						\begin{align}
%							U\ix{GS}=\frac{k\ix{B}T\ix{e}}{e}\ln\left[\mathbf{J}\ix{0}\left(\frac{eU\ix{rf}}{k\ix{B}T\ix{e}}\right)\right] \label{eq:selfbiasdrei}
%						\end{align}
%
%					In \autoref{img:imagundreal} ist der Spannungsverlauf beispielhaft dargestellt. Der \tilt{self-bias} verschwindet demnach nie für Elektrodenspannungen  $U\ix{rf}\neq0$ und ist damit eine feste Größe in Radiofrequenz-Entladungen, welche über kapazitive Bauteile betrieben werden.

  \section{Grenzschichten einer Entladung}\label{sub:rand}

    Im Hauptvolumen eines Plasmas werden die Neutralgasatome durch Wechselwirkungen mit den Elektronen zum Leuchten angeregt. Die näheren Umgebungen von Objekten in Plasmen sind jedoch dunkler, als die eigentliche Entladung. Dies ist auf einen Elektronenmangel zur\"uckruf\"uhren, ähnlich wie in Abschnitt \ref{subsub:verschieb} argumentiert. Jedoch können auch Bereiche mit verschwindenden Elektronendichten hell leuchten, da der sog. \tilt{Glow} auch von der kinetischen Energie und Stoßeffizienz der Ladungsträger abhängt. Die Quasineutralit\"at gilt in der Randschicht nicht mehr. Es folgt, dass diese eine positive Raumladungszone ist. \\
    Auf Grund der wesentlich gr\"o{\ss}eren Beweglichtkeit $\upmu\ix{e}$ und thermischen Geschwindigkeit $v\ix{th,e}$, wird eine Wand in einem Plasma h\"aufiger von Elektronen getroffen, als von den korrespondierenden Ionen. Betrachtet man nur die Oberfl\"ache, so kann man dessen Aufladung und damit auch das Potential $\Phi$ als negativ annehmen.

  \subsection{Child-Langmuir-Gesetz} \label{subsub:childlang}

  Da bei einem kapazitiv gekoppelten \tilt{rf}-Plasma mit asymmetrischen Elektroden \tilt{self-bias} und Verschiebungsstrom nicht-verschwindende Gleichanteile darstellen, kann das \tilt{Child-Langmuir}-Gesetz in Bezug auf diese Größen formuliert werden. Der alternierende Radiofrequenz-Anteil wird vernachlässigt.\\
  Die negative Aufladung einer Wand bei $x=0$ in einem Plasma soll nun eine große Potentialbarriere gegen thermische Elektronen erzeugen, e.g. $|\Phi\left(0\right)-\Phi\left(-d\right)|\ll k\ix{B}T\ix{e}/e\,$. Die Randschicht habe dabei die Dicke $d$. Die Betrachtung in einer Dimension soll hier genügen. Die Elektronendichte $n\ix{e}\left(x\right)$ geht mit dem Boltzmann-Faktor $f\ix{B}\left(\Phi\right)$ ("`\tilt{boltzmann-artige}"' Elektronen) nach \cite{Piel10} wie

    \begin{align}
      n\ix{e}\left(x\right)=n\ix{e}\left(-d\right)f\ix{B}\left(\Phi\right)=n\ix{e}\left(-d\right)\exp\left(\frac{e\left(\Phi\left(x\right)-\Phi\left(-d\right)\right)}{k\ix{B}T\ix{e}}\right) \, . \label{eq:randschichtdichte}
    \end{align}

  Die Elektronendichte fällt damit exponentiell in Richtung der Wand ab. Das bedeutet, dass nur noch Ionen ungehindert einströmen können. Es ist anzunehmen, dass die Randschichtausdehnung $d\ll\lambda\ix{mfp}$ die mittlere freie Weglänge des Plasma ist und die Ionen stoßfrei darin eintreten.\\
  An der Grenze zur Vorschicht (siehe \autoref{img:dichterand}) haben die Ionen die Geschwindigkeit $v\ix{I,0}$ und das Potential der Wand verschwindet gerade an dieser Stelle. Für den Dichteverlauf der Ionen folgt:

    \begin{align}
      n\ix{I}\left(x\right)=n\ix{I}\left(-d\right)\left(1-\frac{2e\Phi\left(x\right)}{m\ix{I}v\ix{I,0}^2}\right)^{\frac{1}{2}}
    \end{align}

  Nimmt man weiterhin an, dass die kinetische Energie $m\ix{I}v\ix{I,0}^2/2\ll |e\Phi\left(x\right)|$ die Beschleunigung in der Randschicht ist,  so ergibt sich die Bestimmungsgleichung (\ref{eq:potential}) für $\Phi\left(x\right)$ nach Poisson. Um dabei das korrekte Potential in der Grenzschicht zu erhalten, muss die Rückwirkung der Ionen auf diese beachtet werden.

    \begin{align}
      \Delta\Phi\cong-\frac{en\ix{I}\left(-d\right)}{\varepsilon\ix{0}}\left(-\frac{2e\Phi\left(x\right)}{m\ix{I}v\ix{I,0}^2}\right)^{-\frac{1}{2}} \label{eq:potential}
    \end{align}

  Die klassische Lösung nach \tilt{Langmuir} für das eindimensionale $\Phi\left(x\right)$ erhält man aus \autoref{eq:potential}, wobei der ungestörte Ionenstrom geschrieben wurde als $j\ix{I}=n\ix{I}\left(-d\right)ev\ix{I,0}$.

    \begin{align}
      \Phi\left(x\right)=\left(\left(\frac{3}{4}\left(x+d\right)\right)^4\left(\frac{j\ix{I}}{\varepsilon\ix{0}}\right)^2\frac{m\ix{I}}{2e}\right)^{\frac{1}{3}}  \label{eq:langmuirpot}
    \end{align}

  Die Auflösung von \autoref{eq:langmuirpot} nach dem Ionenstrom $j\ix{I}$ ergibt das \tilt{Child-Langmuir-Gesetz} (\autoref{eq:childlang}). Dieses gibt nunmehr an, wie der Ionenstrom in der Randschicht von den Eigenschaften des ungestörten Plasmas abhängt. Umgekehrt bedeutet dies, dass die Grenzschicht auf Variationen des Potentials und der Parameter mit Veränderungen der Schichtdicke reagiert und damit versucht, den Einschränkungen der positiven Raumladungszone und dessen Ionenstrom nach dem \tilt{Child-Langmuir-Gesetz} zu genügen.

    \begin{align}
      j\ix{I}=\frac{4}{9}\varepsilon\ix{0}\left(\frac{2e\left(\Phi\left(-d\right)-\Phi\left(0\right)\right)^3}{m\ix{I}d^2}\right)^{\frac{1}{2}} \label{eq:childlang}
    \end{align}

      \begin{figure}
        \centering
        \begin{subfigure}[b]{0.6\textwidth}
          \centering
          \includegraphics[width=0.7\textwidth,height=0.5\textwidth]{figs/randschichtpiel.png}
          \caption{}
          \label{img:dichterand}
        \end{subfigure}
        \begin{subfigure}[b]{0.375\textwidth}
          \centering
          \includegraphics[width=0.8\textwidth,height=0.8\textwidth]{figs/parabelpiel.png}
          \caption{}
          \label{img:parab}
        \end{subfigure}
        \caption{\fett{(a)}: Dichten-Verlauf in der Grenzschicht zu einer Metalloberfläche. In einer bestimmten Entfernung $d$ zur Wand fällt die Elektronendichte praktisch auf 0, woraus die lokale Aufhebung der Quasineutralität folgt. \fett{(b)}: (Eindimensionales) Harmonisches Potential mit extremaler Instabilität. Kleine Auslenkungen sorgen für große Kräfte (nach \cite{Piel10}).}
      \end{figure}

  \subsection{Bohm-Kriterien}

    In Abschnitt (\ref{subsub:childlang}) wurde das Verhalten der Dichten der Ladungstr\"ager in einer Grenzschicht diskutiert. Diese erf\"ullen, in einem Abstand $d$ zu einer Wand in einem Plasma, die Quasineutralit\"atsbedingung nicht mehr. Bisher gab es jedoch noch kein Kriterium, welches die Elektronen daran hindert auch aus weiten Teilen des Plasmas auf die Wand einzustr\"omen. Man stellt sich somit die Frage: Warum dehnt sich die Randschicht nicht in die gesamte Entladung aus? Wieso ist der Bereich der Elektronenersch\"opfung f\"ur eine Kombination der Plasmaparameter konstant? \\
    Um dieses Problem zu L\"osen, stellt man sich ein (mechanisches) Ein-Teilchen-Problem vor (\cite{Piel10}), bei welchem eine Gleichgewichtsanalyse vorgenommen wird. Dabei habe das Potential ein Extremum - maximal oder minimal - an welchem sich eine Punktmasse sich befindet. In diesem Fall sind nur Potentiale mit umgekehrt-parabelartigen Maxima interessant (siehe \autoref{img:parab}). Aus einer kleinen St\"orung folgt eine gro{\ss}e Kraft auf das Teilchen. Es handelt sich um ein instabiles Gleichgewicht.\\
    Der Bezug zum Plasma wird deutlich, wenn man die Differentialgleichung des mechanischen Problems mit der Poisson-Gleichung des Potentials in der Rand- bzw. Vorschicht - siehe \autoref{eq:pseudo}) vergleicht. Die Ladungsverteilung ist durch $\rho$ gekennzeichnet. Die Größe $\Psi\left(\Phi\right)$ stellt ein Pseudopotential dar, welches in seiner Bedeutung mit der mechanischen Variante $V\left(\vec{r}\right)$ vergleichbar ist.

      \begin{align}
        m\frac{\diff^{2}\vec{r}}{\diff t^{2}}=-\frac{\diff V}{\diff\vec{r}} \quad \Leftrightarrow \quad \Delta_{\vec{r}}\,\Phi=-\frac{\diff\Psi}{\diff\Phi}=f\left(\Phi\right)\hspace{-0.3cm}\overset{Poisson}{\overset{\mid}{=}}\hspace{-0.3cm}\frac{\rho}{\varepsilon\ix{0}} \label{eq:pseudo}
      \end{align}

    Damit eine Instabilit\"at vorliegen kann, muss die, aus einer St\"orung des Pseudopotentials resultierende Kraft mit der Entfernung zum Gleichgewicht anwachsen. Die mathematisch \"aquivalente Formulierung ist die Ungleichung in (\ref{eq:beding}). Nach \autoref{eq:pseudo} und der, in Abschnitt (\ref{subsub:childlang}) hergeleiteten Dichten in der Grenzschicht, kann die geforderte Bedingung \"uberpr\"uft werden. Aus ihr folgt das erste \tilt{Bohm-Kriterium}.

      \begin{align}
        0>\left.\frac{\diff^{2}\Psi}{\diff\Phi^{2}}\right|_{\Phi=0}\overset{\autoref{eq:pseudo}}{\overset{\mid}{=}}\left.\frac{\diff}{\diff\Phi}\left(\frac{n\ix{e}\left(x\right)-n\ix{I}\left(x\right)}{\varepsilon\ix{0}}\right)\right|_{\Phi=0}&=\frac{en\ix{e}\left(-d\right)}{\varepsilon\ix{0}}\left(\frac{e}{k\ix{b}T\ix{e}}-\frac{e}{m\ix{I}v\ix{I,0}^{2}}\right) \label{eq:beding} \\
        \Rightarrow \quad v\ix{I,0}\ge v\ix{B,I}=\sqrt{\frac{k\ix{B}T\ix{e}}{m\ix{I}}}& \label{eq:bohmkriteins}
      \end{align}


    Für das Bohm-Kriterium in \autoref{eq:bohmkriteins} kann auch analog die sog. \tilt{Machzahl} $M=v\ix{I,0}/v\ix{B,I}$ angegeben werden, wobei $v\ix{B,I}$ die Bohm-Geschwindigkeit der Ionen ist.\\
    Um zu erläutern, warum sich die Randschicht nicht in das gesamte Plasma ausdehnt, betrachten wir die Teilchenbewegung aus dessen nächster Umgebung. Hier existiert ein elektrisches Feld, welches die Ionen auf die Geschwindigkeit $v\ix{B}$ in Richtung der Wand beschleunigt. Außerdem gilt in ihr noch die Quasineutralitätsbedingung:

      \begin{align}
        n\ix{I}\left(x\right)=n\ix{I,0}\exp\left(\frac{e\Phi\left(x\right)}{k\ix{B}T\ix{e}}\right)=n\ix{e}\left(x\right)\,\, . \label{eq:bohmquasineutral}
      \end{align}

    Hierbei ist $\Phi\left(x\right)$ das Potential der Vorschicht aus Abschnitt (\ref{subsub:childlang}) und $n\ix{I,0}$ die ungestörte Ionendichte des \tilt{bulk}. Da Stöße der Frequenz $\nu\ix{N,I}$ der Ionen mit den Neutralgasatomen einen nicht vernachlässigbaren Einfluss auf deren Strom haben, muss die Geschwindigkeitsverteilung umgeschrieben werden.

      \begin{align}
        \frac{\diff v\ix{I}}{\diff x}=\frac{\nu\ix{N,I}v\ix{I}^2}{v\ix{B}^2-v\ix{I}^2} \label{eq:geschwindverteil}
      \end{align}

    Aus der Singularität von \autoref{eq:geschwindverteil} in $v\ix{I}=v\ix{B}$ und der Kenntnis über das Wandpotential lässt sich die Ausdehnung der Randschicht $d$ bestimmen. Offensichtlich werden Ionen mit einer Geschwindigkeit $v\ix{I}<v\ix{B}$ in der Vorschicht beschleunigt. Geschwindigkeiten größer als die Bohm-Geschwindigkeit kommen dort jedoch nicht vor, da dies nach Gl (\ref{eq:bohmkriteins}) nur in der Randschicht der Fall sein darf. \\
    Zusammen mit dem ersten \tilt{Bohm-Kriterium} folgt, dass am Übergang der Vor- zur Randschicht die Ionen $v\ix{B}$ erreichen müssen, damit sich eine positive Raumladungszone ausbilden kann.

      \begin{align}
        M=1 \quad \Leftrightarrow \quad v\ix{I}\left(-d\right)=v\ix{B} \label{eq:bohmkritzwei}
      \end{align}

    Lokal heißt das, dass bei $x=-d$ die Dichten bereits auf $\approx0,66n\ix{e,0}$ abgefallen sind (siehe \ref{eq:bohmquasineutral}, \autoref{img:dichterand}) und das Potential durch die Aufladung der Wand circa $-k\ix{B}T\ix{e}/2e$ beträgt.\\
    Ein Plasma "`sieht"' damit seine Randschicht nicht, da sich die notwendige Dynamik der entsprechenden Ladungsträger auf diese beschränkt und sich nicht beliebig ausdehnen kann. Mit anderen Worten: eine Randschicht bildet sich nur an Orten des Elektronenmangels und externer, negativer Potentiale aus.

  \section{Aufladung von Staubpartikeln}\label{sub:ströme}

    Die Ladung eines Fremdteilchens (Staub) in einem Plasma ist eine dynamische Größe. Sie ist sowohl zeitlich veränderlich, als auch abhängig von den Plasmaparametern, den Partikeleigenschaften sowie deren Trajektorie in der Entladung. Die Ladung eines Teilchens zum Zeitpunkt $t$ ergibt sich, näherungsweise, aus den Ladungsströmen $I\ix{k}$ der Plasmaspezies auf das Partikel bis zu diesem Zeitpunkt. Im folgenden genügt es, dieses Problem auf einer Zeitskala zu betrachten, in der man die Ladung als konstant unter dem Einfluss der Ströme annehmen kann. Damit werden diese ebenfalls stationär und es gilt für ein einzelnes Teilchens die \tilt{Kirchhoff'sche Knotenregel}, wobei der Staub ein Knoten sei:

      \begin{align}
        \sum_{\text{k}} I\ix{k}\left(\Phi\ix{fl}\right)=\frac{\diff Q\ix{S}}{\diff t} \,\, . \label{eq:float}
      \end{align}

  Die Elektronen und Ionen im Plasma strömen aufgrund ihrer thermischen Bewegung auf das Fremdteilchen und verleihen diesem über Stöße eine Ladung $Q\ix{S}$, wobei sich für das Partikel ein elektrostatisches Potential $\Phi\ix{fl}$, das sog. \tilt{floating} Potential, einstellt, für welches \autoref{eq:float} gilt. Die Ladungsspezies können aus Sekundär-,Photo-, oder Feldemissionen kommen, wobei die dominanten Ströme die des Plasmas selbst sind. Hier soll es ausreichen, die Plasmaströme nach  \tilt{Langmuir} und \tilt{Mott-Smith} mit dem sog. \tilt{orbital motion limit}-Modell \cite{Langmuir26} zu beschreiben. 

    \subsection{OML-Modell}\label{subsub:oml}

    Dabei wird angenommen, dass sich ein strömendes Teilchen, welches zu $Q\ix{S}$ beiträgt bzw. damit elektrostatisch wechselwirkt, stoßfrei aus dem Unendlichen (durch die Entladung) darauf zu bewegen kann. Aufgrund der, im Allgemeinen höheren Elektronentemperatur und -beweglichkeit wird $\Phi\ix{fl}<0$, woraus eine veränderte Wechselwirkung mit den Teilchen der Ladungsströme folgt. Man definiert folglich einen kritischen Stoßparameter $b\ix{k}$, welcher dem Abstand eines Ions zur Target-Achse entspricht, bei dem dieses durch die Coulomb-Anziehung des Partikels gerade noch dessen Oberfläche tangiert. Für Parameter $b>b\ix{k}$ wird ein eintreffendes Ion auf seiner Trajektorie nur abgelenkt, für $b<b\ix{k}$ landet dieses auf dem Staubteilchen, siehe \autoref{img:oml}.\\
    Da vorausgesetzt wurde, dass keine Stöße vor dem Target mit dem Radius $a$ geschehen, kann die Impulserhaltung zusammen mit der Energieerhaltung für ein einzelnes Projektil aufgestellt werden, welches sich im Abstand $b\ix{k}$ darauf zu bewegt (hier aus \cite{Melzer12}):

      \begin{align}
        |\vec{L}|=|\vec{r}\times\vec{p}|=m\ix{I}v\ix{I}a=m\ix{i}v\ix{i,0}b\ix{k} \label{eq:impulserhaltung} \\
        E\ix{I}=\frac{m\ix{I}}{2}v\ix{I}^2+e\Phi\ix{fl}=\frac{m\ix{I}}{2}v\ix{I,0}^2 \label{eq:energieerhaltung}
      \end{align}

    In Gl.(\ref{eq:impulserhaltung}) und (\ref{eq:energieerhaltung}) stehen jeweils die linken Seiten für den Zeitpunkt des Auftreffens und die rechten für das Einlaufen aus dem Unendlichen. Durch Umformungen lässt sich ein Ausdruck für den kritischen Stoßparameter, in Abhängigkeit vom Partikelradius und dem \tilt{floating}-Potential aufgestellt werden.

          \begin{figure}[!t]
            \centering
            \includegraphics[width=0.7\textwidth, height=0.3\textwidth]{figs/orbitalmotionlimitmelzer.png}
            \caption{OML-Beschreibung eines, auf ein Staubpartikel einströmendes Ions. Der kritische Stoßparameter, für welcher eine Kollision stattfindet, ist $b\ix{k}$. Er ist im Vergleich zum geometrischen Querschnitt aufgrund der Coulomb-Wechselwirkung vergrößert (nach \cite{Melzer12}).}
            \label{img:oml}
          \end{figure}

      \begin{align}
        b\ix{k}^2=a^2\left(1-\frac{e\Phi\ix{fl}}{\frac{m\ix{I}}{2}v\ix{I,0}^2}\right) \label{eq:krit}
      \end{align}

    Der zugehörige Streuquerschnitt für die Streuung eines Teilchenstroms an einem Staubpartikel wird damit zu $\sigma\ix{k}=\pi b\ix{k}^2$, welcher größer als die geometrische Querschnittsfläche $\pi a^2$ ist. Dem liegt die Coulomb-Anziehung der Stoßpartner zu Grunde.\\
    Der (differentielle) Ladungsträgerstrom $\diff I\ix{j}$, welcher schließlich die Partikelladung bzw. -potential $\Phi\ix{P}$ bestimmt, ergibt sich aus der Aufsummierung aller Stromdichten von Teilchen der Geschwindigkeiten $v\ix{j}$, gewichtet mit ihren zugehörigen Wechselwirkungsquerschnitten $\sigma\ix{j}$ und (hier: maxwellartigen) Verteilungen $f\left(v\ix{j}\right)$ (\cite{Melzer12}) der Geschwindigkeiten:

      \begin{align}
        \diff I\ix{j}=\sigma\ix{j}&\left(v\ix{j}\right)n\ix{j}v\ix{j}f\left(v\ix{j}\right)\diff v\ix{j} \nonumber \\
        \text{Ionen: } I\ix{I}&=\pi a^2n\ix{I}e\sqrt{\frac{8k\ix{B}T\ix{I}}{\pi m\ix{I}}}\left(1-\frac{e\Phi\ix{P}}{k\ix{B}T\ix{I}}\right) \label{eq:ionenstrom} \\
        \text{Elektronen: } I\ix{e}&=-\pi a^2n\ix{e}e\sqrt{\frac{8k\ix{B}T\ix{e}}{\pi m\ix{e}}}\exp\left(\frac{e\Phi\ix{P}}{k\ix{B}T\ix{e}}\right) \label{eq:elektronenstrom}
      \end{align}

    Der Unterschied zwischen \autoref{eq:ionenstrom} und \autoref{eq:elektronenstrom} resultiert aus den unterschiedlichen Arten der Wechselwirkung mit dem Staubteilchen. Da $\Phi\ix{P}<0$ ist, werden Ionen aller Geschwindigkeiten in Richtung des Partikels gelenkt und könnten theoretisch damit stoßen (mit Rücksicht auf die Geschwindigkeitsverteilung und den Streuquerschnitt). Die Elektronen hingegen müssen mindestens eine Geschwindigkeit $v\ix{min}=\sqrt{-2e\Phi\ix{P}/m\ix{e}}$ besitzen, damit sie die Potentialbarriere zum Partikel überwinden und damit stoßen können. Au{\ss}erdem finden sich mit $\sqrt{8k\ix{B}T\ix{j}/\pi m\ix{j}}$ die thermischen Geschwindigkeiten der jeweiligen Spezies $j$ als Vorfaktoren wieder. Damit werden die Gesamtstr\"ome letztendlich zum Produkt aus ungest\"orter, thermischer Stromdichte und der angepassten Wechselwirkungsfl\"ache des Partikels. \\
    Es sei erwähnt, dass die OML-Theorie nicht der Realität entspricht. Die mittlere freie Weglänge eines Ions oder Elektrons hat in etwa die Dimension der Debyelänge $\lambda\ix{D}$ und ist nicht, wie vorausgesetzt, unendlich groß. Des weiteren entsprechen die $f\left(v\ix{j}\right)$ in der Praxis nicht isotropen Maxwell-Geschwindigkeitsverteilungen.

    \subsection{Neutralgas-Ionen-Stöße}

    Auf Grundlage der OML-Theorie lässt sich ein erweiterter Ausdruck für den Ionenstrom mit Rücksicht auf Stöße des Neutralgases aufstellen. Offensichtlich geht, mit dem Durchgang eines Projektils durch das Neutralgas einer Entladung, der Verlust kinetischer Energie einher. Es ist nun leicht nachzuvollziehen, dass sich Ionen, aufgrund der Ablenkungen durch Stöße und Wechselwirkung, um das negativ geladene Staubpartikel sammeln (\cite{Goree92}).\\
          In einer Sphäre mit dem Radius $R$ um das Teilchen ist die Stoßwahrscheinlichkeit zwischen Ionen und Neutralgasatomen proportional zu

      \begin{align}
        \frac{R}{\lambda\ix{mfp}}=Rn\ix{N}\sigma\ix{I,N} \,\, . \label{eq:wahrschein}
      \end{align}

    ($n\ix{N}$ - Neutralgasdichte; $\sigma\ix{I,N}$ - Stoßparameter für Ion-Neutralgas-Stöße)\\
    Durch eine Sph\"are mit diesem Radius flie{\ss}t nun neben dem thermischen Ionenstrom $I\ix{th}$ noch ein weiterer Strom $I\ix{S}$ auf Grund der St\"o{\ss}e mit dem Neutralgas. Die Ionen der Geschwindigkeit $v\ix{th,I}$, welche sich ohne Wechselwirkung mit dem Gas der Entladung am Target vorbei bewegen w\"urden, werden mit der Sto{\ss}wahrscheinlichkeit aus \autoref{eq:wahrschein} in Richtung des Staubteilchens abgelenkt und tragen somit zum Ladungsstrom auf dieses bei. Der Gesamtladungsstrom durch Ionen entspricht nach \cite{Lampe01} der Summe der beiden Str\"ome:

      \begin{align}
        I\ix{th}=&\pi R^{2}n\ix{N}ev\ix{th,I} \quad I\ix{S}=\pi R^2n\ix{N}ev\ix{th,I}\left(Rn\ix{N}\sigma\ix{N,I}\right) \\
        &I\ix{ges}=\pi a^{2}n\ix{I}ev\ix{th,I}\left(1-\frac{e\Phi\ix{P}}{k\ix{B}T\ix{I}}+\frac{R^{3}}{\lambda\ix{mfp}a^{3}}\right)
      \end{align}

    Auf der Oberfl\"ache der '\tilt{Sto{\ss}sph\"are}' $\mathbb{S}\ix{R}$ hat das \tilt{Yukawa}-Potential des Staubteilchens der Ladungszahl $Z\ix{S}$ gerade die Energie der thermischen Ionenbewegung (\autoref{eq:bestimmR}), womit diese gerade noch in dessen Einfang geraten. Mit dem daraus bestimmten $R$ folgt nach (\cite{Melzer12},\cite{Khrapak05a}) ein diskreter Ionenstrom auf das Partikel mit R\"ucksicht auf die Ionen-Neutralgaswechselwirkung (\autoref{eq:ionstromkorr}).

      \begin{align}
        \frac{k\ix{B}T\ix{I}}{e}=&\frac{Z\ix{S}e}{4\pi\varepsilon\ix{0}R}\exp\left(\frac{R}{\lambda\ix{D}}\right) \label{eq:bestimmR} \\
        I\ix{I}=\pi a^{2} n\ix{I}e v\ix{th,I}&\left(1-\frac{e\Phi\ix{P}}{k\ix{B}T\ix{I}}+0,1\left(\frac{e\Phi\ix{P}}{k\ix{B}T\ix{I}}\right)^{2}\frac{\lambda\ix{D}}{\lambda\ix{mfp}}\right) \label{eq:ionstromkorr}
      \end{align}

    Im Vergleich zum Ergebnis des OML-Modells aus \autoref{eq:ionenstrom} ist dieser Ladungstrom in Abh\"angikeit des Partikelpotentials und Ionentemperatur vergr\"o{\ss}ert. Die Wechselwirkung mit dem Neutralgas sorgt also, \"uber die St\"o{\ss}e und Erzeugung einer \tilt{Ionenwolke} um ein Teilchen, f\"ur einen gr\"o{\ss}eren pos. Ladungsstrom, welcher respektive wiederum f\"ur eine Steigerung des Elektronenstromes auf Grund des ver\"anderten Partikelpotentials sorgt. Eine selbstkonsistente L\"osung der, an den Anfang gestellte Problematik in \autoref{eq:float} um das \tilt{floating}-Potential gestaltet sich jedoch als \"uberaus schwierig, da schon der OML-Theorie mehrere idealisierte Annahmen vorausgingen und dessen Eigenr\"uckwirkungen nicht-trivial sind.\\
    Um abschließend die Staubladung zu bestimmen, kann man das Modell eines Kugelkondensators der Kapazität $C\ix{S}$ hernehmen und dieses mit einer Abschirmung $a/\lambda\ix{D}$ verknüpfen (nach \cite{Melzer12}).

      \begin{align}
        Q\ix{S}=Z\ix{S}e=C\ix{S}\Phi\ix{fl}=4\pi\varepsilon\ix{0}a\left(1+\frac{a}{\lambda\ix{D}}\right)\Phi\ix{fl} \label{eq:ladung}
      \end{align}

    Für typische Laborplasmen kann das \tilt{floating}-Potential zu $\Phi\ix{fl}\approx-2k\ix{B}T\ix{e}/e$ genähert werden, woraus die Ladungszahl sich zu $Z\ix{S}\approx1400\cdot a/\unit[1]{\upmu m}\cdot T\ix{e}/\unit[1]{eV}$ ergibt.

  \section{Staub-Dynamik}\label{sub:dynamik}

    In einem Plasma wirken viele, u.U. nicht-triviale Kräfte auf den eingefangenen Staub. Im Folgenden werden die wichtigsten Einflüsse und Kenngrößen der Dynamik komplexer Plasmen vorgestellt und beschrieben (nach \cite{Melzer10}). Insbesondere ist dieser Abschnitt wichtig für das Verständnis über die Bildung und Stabilität sog. \tilt{Yukawa-Cluster}. Dabei handelt es sich um ein System aus wenigen Staubteilchen, welche sich in einem äußeren, harmonischen Potential auf konzentrischen Kugelschalen anordnen. Diese sollen des Weiteren genauer in Abschnitt \ref{sub:yukawaclust} beschrieben werden.

    \subsection{Gravitation und elektrische Feldstärke}\label{subsub:grav}

    Betrachtet man ein Experiment, welches am Erdboden durchgeführt wird, so muss die vollständige Gravitationskraft berücksichtigt werden. Dies gilt beispielsweise nicht für Versuche unter Mikrogravitation während Parabelflügen oder auf der \tilt{International Space Station}.

      \begin{align}
        F\ix{g}=m\ix{S} g=\frac{4}{3}\pi a^3 \rho\ix{S} g
      \end{align}

    ($m\ix{S}$ - Masse der Staubteilchen; $\rho\ix{S}$ - Massendichte des Staubes; $g$ - Erdbeschleunigung)\\
    Natürlich wirkt auf die, durch das ionisierte Gas elektrisch geladenen Partikel eine elektrische Kraft $F\ix{E}$, welche aus dem äußeren Feld $E$ der \tilt{rf}-Elektroden folgt. Eine Kraft durch das Feld des lokalen Plasmas tritt nicht auf: innerhalb einer \tilt{Debye-Kugel} sind die Veränderung zu schnell, als dass das träge Staubteilchen diesen folgen könnte. 

      \begin{align}
      F\ix{E}=Q\ix{S} E=4 \pi \varepsilon\ix{0} a \Phi\ix{fl} E
      \end{align}

    Diese beiden Kräfte heben sich gerade in der Randschicht einer Radiofrequenz-Entladung auf, da sie antiparallel zueinander stehen. Zu beachten ist hierbei der stark unterschiedliche Einfluss des Teilchenradius - $F\ix{g}\propto a^3$ und $F\ix{E}\propto a$.\\

    \subsection{Abschirmung und Polarisationskräfte}\label{subsub:abschirm}

    Die große negative Aufladung der Staubteilchen sorgt über die Coulomb-Wechselwirkung mit den auf das Partikel zuströmenden Ionen dafür, das sich die Konzentration derer lokal stark ändert. Es entsteht eine Wolke aus langsamen Ionen die quasi in der näheren Umgebung um das Teilchen verbleiben, jedoch nicht mit diesem interagiert und es nach außen hin vor dem Einfall schnellerer pos. Ladungen abschirmen. Somit gibt es keine direkte Rückwirkung der Wolke auf das Partikel, sofern dessen sphärische Symmetrie gegeben ist. Gilt dies nicht, so entsteht ein Multipol- bzw. Dipolmoment $\vec{p}$, welches danach strebt, sich in Richtung des Feldes $\vec{E}$ auszurichten. Damit wirkt eine Kraft $F\ix{Dip}$ (für ein Dipolmoment) auf das Staubteilchen zurück, welche mit dem Gradienten der Richtungsdifferenz zwischen $\vec{p}$ und $\vec{E}$ geht.

      \begin{align}
        \centering
        \vec{F}\ix{Dip}&=\vec{\nabla}\left(\vec{p}\vec{E}\right)\\
        &{\overset{\vec{p}\,||\vec{E}}{=}}\grad{pE} \nonumber
      \end{align}

    Das besagte Dipolmoment entsteht u.a. durch die diversen, gerichteten Ladungsprozesse in dem Plasma. Ein Partikel, welches in der Randschicht eingefangen und von einer Ionenwolke umgeben wird, 'sieht' unterschiedliche \tilt{Debye}-Längen über und unter sich aufgrund der lokalen Feldrichtung und der stark vom Mittelwert abweichenden Ionen- und Elektronendichten. Somit ändert sich offensichtlich die Plasmadichte innerhalb des Volumens einer Kugel mit dem, nun ortsabhängigen Radius $\lambda\ix{D}\left(\vec{r}\right)$. Insgesamt folgt daraus eine neue Bestimmungsgleichung für das Potential und damit auch eine neue Kraft $F\ix{E}$.

      \begin{align}
        \Delta \Phi\left(\vec{r}\right)-\frac{\Phi\left(\vec{r}\right)}{\lambda\ix{D}^2\left(\vec{r}\right)}=\frac{Q\ix{s}}{\varepsilon\ix{0}}\delta\left(\vec{r}\right) \\
        F\ix{E}=\underbrace{Q\ix{S}E}_{\text{(I)}}-\underbrace{\frac{Q\ix{S}^2}{8\pi\varepsilon\ix{0}}\frac{\nabla\lambda\ix{D}\left(\vec{r}\right)}{\lambda\ix{D}^2}}_{\text{(II)}}
      \end{align}

    Hierbei stellt (I) die normale Komponente dar und (II) ist zusätzliche Kraft durch die Deformation der Ionenwolke in Richtung kleinerer \tilt{Debye}-Längen $\lambda\ix{D}\left(\vec{r}\right)$. Die Kraft $F\ix{E}$ kann also dadurch größer oder kleiner werden. Hinzu kommt, dass der veränderte Parameter $\lambda\ix{D}$ von der Driftgeschwindigkeit $u\ix{I}$ abhängt, welche die Ionenwolke maßgeblich beeinflusst. Ist jedoch $u\ix{I}<v\ix{th,I}$ so kann man annehmen, dass die Ionenwolke besteht und $\lambda\ix{D}\left(\vec{r}\right)\approx\lambda\ix{D,I}$ gilt.\\
    Sind die Ionen jedoch schneller, womit $u\ix{I}>>v\ix{th,I}$ wird, so können sie nicht mehr vom Feld des Partikels 'gefangen' werden und damit die Ionenwolke bilden. So gilt folglich $\lambda\ix{D}\left(\vec{r}\right)\approx\lambda\ix{D,e}$.\\
    Insgesamt ist der Einfluss der Polarisation vernachlässigbar klein, solange die Teilchen eine Größe von einigen hundert $\unit{\upmu m}$ nicht übersteigen.

  \subsection{Ionen-Reibung}\label{subsub:reibung}

    Aufgrund der hohen negativen Ladung des Staubteilchens und der daraus resultierenden elektrostatischen Wechselwirkung existiert ein, relativ auf das Partikel zufließender, Ionenstrom oder auch "`Ionenwind"'. Weiterhin bewegen sich Neutralgasatome durch das Plasma und stoßen mit anderen Teilchen. Die Zahl der Stöße $\diff N$ einer Spezies mit einem Target mit dem Streuparameter $\sigma$ im Zeitintervall $\diff t$ ist somit über deren relative Geschwindigkeit $v\ix{rel}$ zu diesem in $\diff N=n\sigma v\ix{rel}\diff t$ gegeben. Hierbei kennzeichnet $n$ die Stromdichte der strömenden Teilchen. Aus diesem Strom folgt ein gewisser Impulsübertrag $\Delta p$ auf das Target. Hieraus lässt sich eine Kraft $F\ix{drag}$ ziehen, die einer Reibung bzw. einer Impulsaufnahme entspricht.

      \begin{align}
        F\ix{drag}=\frac{\diff N \Delta p}{\diff t}=\Delta p n \sigma v\ix{rel}
      \end{align}

     Man kann diese aus 2 Teilen aufbauen: direkte Kollisionen mit den Ionen $F\ix{dir}$ und deren Coulomb-Streuung an den Feldern der neg. geladenen Staubpartikel. Im folgenden soll das sog. \tilt{Barnes-Modell} eingeführt werden, welches die besagte Ionenreibung beschreibt.

      \paragraph{Ionenreibung: \tilt{Barnes-Modell}}

      Für die Bestimmung von $F\ix{dir}$ wird angenommen, dass nur die Ionen, welche für eine Ladungsänderung der Partikel sorgen, auch diese direkt treffen. Im Rückblick auf die Bestimmung der Ionenströme in \ref{sub:ströme} wird die Kraft durch Kollisionen zu

        \begin{align}
          F\ix{dir}=\pi a^2m\ix{I}\tilde{v}n\ix{I}u\ix{I}\left(1-\frac{2e\Phi\ix{fl}}{m\ix{I}\tilde{v}^2}\right)\,\,.
        \end{align}


      Das Produkt aus Masse und mittlerer Geschwindigkeit der Ionen $m\ix{I}\tilde{v}=m\ix{I}\sqrt{u\ix{I}^2+v\ix{th,I}^2}$ entspricht dem Impulsübertrag ($u\ix{I}$ - Ionen-Driftgeschwindigkeit).\\
      Für die Coulomb-Streuung müssen alle diejenigen Ionen miteinbezogen werden, welche mit dem Feld des Staubes 'stoßen'. Hierfür wird der Streuquerschnitt $\tilde{\sigma}$ nach \cite{Barnes92} der Ionen-Elektronen-Wechselwirkung auf den aktuellen Fall angepasst.

        \begin{align}
          \tilde{\sigma}=4\pi b_{\frac{\pi}{2}}^2\ln\left(\Lambda\right)=&\,4\pi b_{\frac{\pi}{2}}^2\ln\left(\frac{\lambda\ix{D}}{b_{\frac{\pi}{2}}}\right) \\
          \text{wobei }\, b_{\frac{\pi}{2}}=&\,\frac{e^2}{4\pi\varepsilon\ix{0}m\ix{I}v^2} \nonumber
        \end{align}

      Der Stoßparameter $b_{\frac{\pi}{2}}$ beschreibt eine Ablenkung um $\unit[90]{\degree}$. Für die Ionenreibung müssen weitere Bedingungen mit eingebunden werden: die Coulomb-Streuung außerhalb der Ionenwolke ist irrelevant, die Staubpartikel haben eine endliche Ausdehnung und damit existiert ein minimaler Stoßparameter $b\ix{k}$. Damit wird das gesuchte $\sigma$ zu

        \begin{align}
          \sigma=\int_{b\ix{k}}^{\lambda\ix{D}}\tilde{\sigma}\diff\left(\frac{\lambda\ix{D}}{b_{\frac{\pi}{2}}}\right)=&\,4\pi b_{\frac{\pi}{2}}^2\ln\left(\frac{\lambda\ix{D}^2+b_{\frac{\pi}{2}}^2}{b\ix{k}^2+b_{\frac{\pi}{2}}^2}\right)^{1/2} \nonumber \\
          F\ix{Coul}\overset{(*)}{=}&\,\frac{2\pi a^2 e^2\Phi\ix{fl}^2}{m\ix{I}\tilde{v}^3}n\ix{I}u\ix{I}\ln\left(\frac{\lambda\ix{D}^2+b_{\frac{\pi}{2}}^2}{b\ix{k}^2+b_{\frac{\pi}{2}}^2}\right)^{1/2} \, \, .
        \end{align}

      Den $\ln\left(\dots\right)$ nennt man auch den \tilt{Coulomb-Logarithmus} $\ln\left(\Lambda\right)$.\\
      In (*) wurde für die vollständige Coulomb-Kraft durch Ionen $F\ix{Coul}$ auf ein Partikel benutzt, dass dieses mit Ladung $Z\ix{S}e=Q\ix{S}$ als Kugelkondensator mit $Q\ix{S}=4\pi\varepsilon\ix{0}a\Phi\ix{fl}$ ausgedrückt werden kann. Wichtig zu beachten ist jedoch, dass nur Ionen einen Beitrag leisten, die innerhalb eines Stoßparameters $\lambda\ix{D}\approx\lambda\ix{D,e}$ (nach \cite{Kilgore93}) am \tilt{target} vorbei fliegen.\\
      Die gesamte Kraft durch Ionen auf die Staubteilchen ist somit die Summe der Kräfte aus direkten Stößen und Coulomb-Kollisionen $F\ix{ion}=F\ix{dir}+F\ix{Coul}$, welche zusammen dargestellt sind in \autoref{img:ionkräfte}.

        \begin{figure}[!t]
          \centering
          \includegraphics[height=0.4\textwidth,width=0.6\textwidth]{figs/forcesandtrappingmelzer.png}
          \caption{Berechnete Kräfte $F\ix{Coul}$ (\tilt{Coulomb}), $F\ix{dir}$ (\tilt{collection}) und die Summe beider auf einer doppelt-logarithmischen Skala(nach \cite{Melzer12}).}
          \label{img:ionkräfte}
        \end{figure}

  \subsection{Neutralgasreibung}

    Die Stöße mit den Neutralgasatomen können, ebenso wie die mit Ionen, als Kraft durch Reibung aufgefasst werden. Diese sorgen insbesondere für eine Verlangsamung der Bewegung der Staubteilchen. Die Kraft $F\ix{N}$ wird, in ähnlicher Weise wie $F\ix{dir}$, durch einen Impulsübertrag über einen Strom von Neutralgasteilchen auf einen effektiven Querschnitt ausgedrückt (\cite{Epstein24}).

      \begin{align}
        F\ix{N}=-\delta\frac{4}{3}\pi a^3m\ix{N}v\ix{th,N}n\ix{N}v\ix{S}
      \end{align}

    ($m\ix{N}$ - Neutralgasatommasse; $v\ix{th,N}$ - thermische Geschw. der Neutralgasatome; $n\ix{N}$ - Neutralgasdichte; $v\ix{S}$ - Staubteilchengeschw.)\\
    Der Faktor $\delta$ beschreibt hierbei die Art, wie die Neutralgasatome mit dem Staub stoßen. Eine spiegelnde Reflexion tritt für ein $\delta=1$ auf. Mit steigendem $\delta$ wird die Kollision immer diffuser, bis hin zu einem Wert von $\delta=1,44$.\\
    Für den berühmten \tilt{Milikan}-Öltropfen-Versuch zur Bestimmung der Ladung eines Elektrons wurde, 1924 von \tilt{P. S. Epstein}, ebenso ein Ausdruck für die Kraft durch Neutralgasreibung bestimmt. Dabei ist $\beta$ der Reibungskoeffizient und $\rho\ix{S}$ die Dichte des Staubmediums.

      \begin{align}
        F\ix{N}=-m\ix{S}\beta v\ix{S}=-m\ix{S}v\ix{S}\delta\frac{8}{\pi}\frac{p}{a\rho\ix{S}v\ix{th,N}}
      \end{align}

  \subsection{Thermophoretische Kraft}\label{subsub:therm}

    In einer Plasmakammer kann, durch Aufheizung oder Abkühlung einer der Elektroden bzw. Kammerbegrenzungen ein, der Gravitation oder dem elektrischen Feld entgegen gerichteter Temperaturgradient angelegt werden.\\
    Die Kraft kann folgendermaßen erklärt werden: auf der Seite der höheren Temperatur haben die Neutralgasteilchen im Mittel eine größere Geschwindigkeit und Impuls, woraus ein positiver Impulsübertrag in Richtung niedrigerer Temperaturen folgt.	Mit der Wärmeleitfähigkeit des Neutralgases $\kappa\ix{N}$ folgt für die thermophoretische Kraft $F\ix{th}$ \autoref{eq:therm}. Auf Grund $\propto a^2$ ist diese Kraft besonders wichtig für Teilchen mit einem Radius kleiner als $\unit{\upmu m}$. Sie wird für die Formation von sog. \tilt{Yukawa-Clustern} (siehe \ref{sub:yukawaclust}) oder zur Herstellung von "`Quasi-Schwerlosigkeit"' (siehe \cite{Rothermel02}) genutzt.

      \begin{align}
        F\ix{th}=-\frac{32}{15}\frac{a^2 \kappa\ix{N}}{v\ix{th,N}}\grad{T} \label{eq:therm}
      \end{align}

    Es sei erwähnt -  über die Zustandsgleichung für ideale Gase $pV=Nk\ix{B}T$ ersichtlich -, dass bei einer höheren Temperatur die Dichte des Neutralgases sinkt. Experimentell findet man daher, dass die verminderte Dichte zu einer niedrigeren Stoßintensität führt und damit den Effekt der Thermophorese in etwa gerade kompensiert. Trotzdem bleibt die Methode des Temperaturgradienten ein wichtiges Mittel zum Einfang des Staubes und Ausgleich der Gravitationskraft.

  \subsection{Kraft durch \tilt{Laser}-Einstrahlung}\label{subsub:laser}

    Der Einsatz von Laser hat in Versuchen zu kolloidalen Plasmen verschiedene Gründe: einerseits wird das Beobachtungsvolumen mit ihnen ausgeleuchtet, andererseits können unterschiedliche dynamische Eigenschaften des Staubes untersucht werden.\\
    Die Manipulation von eingefangenen Teilchen wird bspw. durch die Fokussierung eines \tilt{Laser}strahls auf ein Partikel realisiert. Hierbei spielt der Impulsübertrag der Photonen mit Impuls $p\ix{Ph}$, welcher dem Strahlungsdruck $p\ix{Strahl}$ entspricht, und eine Kraft durch den Laser $F\ix{Strahl}$ \cite{Ashkin70} eine Rolle.

      \begin{align}
        p\ix{Strahl}=\frac{\diff p\ix{ph}}{A\ix{L}\diff t}=&\frac{\diff N\ix{ph}}{A\ix{La}\diff t}\frac{h}{\lambda}=\frac{I\ix{L}}{c} \nonumber \\
        F\ix{Strahl}=&\gamma\frac{I\ix{L}}{c}\pi a^2 \label{eq:strahl}
      \end{align}

    ($N\ix{ph}$ - Anzahl der, die das Partikel treffenden Photonen; $\lambda$ - Wellenlänge; $A\ix{L}$ - Querschnittsfläche des \tilt{Laser}; $\gamma$ - Wechselwirkungskoeffizient für den Stoß Photon-Staub)\\
    Für ein $\gamma=2$ in \autoref{eq:strahl} liegt eine Totalreflexion vor, für $\gamma=1$ wird der Impuls der auftreffenden Photonen vollständig absorbiert. Im Zusammenhang mit dem Versuch dieser Arbeit ist die Kraft durch Laser jedoch nur ein unerwünschter Nebeneffekt - im Sinne einer zusätzlichen Kraft - der Beleuchtung des Yukawa-Balls.\\
    Weiterhin sei angemerkt, dass auf Grund des anisotropen Profils der Photonendichte (\tilt{gaussian bandwidth}) über den Querschnitt eines Laser-Strahls, der Einfang von (einem) ausgewählten Staubpartikeln realisiert werden kann (\tilt{Laserpinzette}).

%			Ein \tilt{Laser}-Strahl kann u.a., auf Grund der anisotropen Photonendichte (\tilt{gaussian bandwidth}) über den Querschnitt, zum Einfang von (einem) ausgewählten Partikeln benutzt werden. Bei der photophoretischen Kraft wird die resultierende Dynamik aus dem, durch die Stöße mit Photonen erzeugten Temperaturgradienten über ein einzelnes Partikel beschrieben. Wie bereits erläutert, liegt ein Strahldichtegradient für eine Fläche von $A\ix{S}$ vor, woraus eine unterschiedliche Staubaufheizung durch die Stoßreibung folgt.\\
%			Treffen nun Neutralgasatome auf die heißere Seite, so werden sie dort schneller reflektiert, als von der kälteren. Die Differenz des Impulsübertrags resultiert in der photophoretischen Kraft, die entgegen des Gradienten der Photonendichte und damit in Richtung der heißen Seite der Partikel zeigt. Je nach der Absorptionsfähigkeit besitzen die Teilchen unterschiedliche Temperaturgradienten. Somit stellt sich eine heiße Front aus den Teilchen auf, welche eine gute Absorption und somit insgesamt höhere Temperatur aufweisen. Für diese Partikel zeigt $F\ix{ph}$ in Richtung des Strahls. Analog wirkt die photophoretische Kraft für schlecht absorbierende Teilchen in entgegengesetzte Richtung. Somit ergibt sich $F\ix{ph}$ in \autoref{eq:photoph} mit dem Wärmeleitungskoeffizienten $\kappa\ix{S}$ des Staubes, dem Gasdruck $p$ und der Gastemperatur $T$.

%				\begin{align}
%					F\ix{ph}=\frac{\pi a^2 p I\ix{L}}{6\left(pav\ix{th,N}+\kappa\ix{S}T\right)} \label{eq:photoph}
%				\end{align}

    \begin{figure}[!t]
      \centering
      \begin{subfigure}[t]{0.49\textwidth}
        \includegraphics[width=\textwidth,height=\textwidth]{figs/allforcesequlibriummelzerlinks.png}
        \caption{}
        \label{img:linkseq}
      \end{subfigure}
      \begin{subfigure}[t]{0.49\textwidth}
        \includegraphics[width=\textwidth,height=\textwidth]{figs/allforcesequlibriummelzerrechts.png}
        \caption{}
        \label{img:rechtseq}
      \end{subfigure}
      \caption{Kräfte als Funktion des Teilchenradius in einem Argon-Plasma (Parameter: $\kappa\ix{N}=\unit[0,016]{\frac{kg m}{s^3}}$;  \mbox{$\rho\ix{S}=\unit[1,5\cdot\tenpo{3}]{\frac{kg}{m^3}}$}; $T\ix{e}=\unit[2]{eV}$; $\Phi\ix{fl}=\unit[-4]{V}$; $E=\unit[1000]{\frac{V}{m}}$; $n\ix{I}=\unit[\tenpo{15}]{m^{-3}}$; $u\ix{I}=v\ix{th,I}=v\ix{th,N}=\unit[400]{\frac{m}{s}}$, nach \cite{Melzer12}) }
      \label{img:kräfte}
    \end{figure}

  \subsection{Einfang und Gleichgewicht}\label{subub:einfang}

    In \autoref{img:kräfte} sind einige der bisher beschriebenen Kräfte für typische Parameter berechnet und mit einer doppelt-logarithmischen Skala dargestellt worden. Für Teilchen mit einem Radius im Bereich von einigen $\unit{\upmu m}$ sind die Kräfte des äußeren elektrischen Feldes $F\ix{E}$ und der Gravitation $F\ix{G}$ dominant. Daher müssen, für einen praktikablen Einfang, diese beiden Kräfte im Gleichgewicht sein, d.h. sich stationär aufheben. Dies ist gerade in der nahen Randschicht der (getriebenen) Elektrode der Fall, da dort $F\ix{E}$ stark genug ist. Weil $\grad{E}/E\ll1$ ist, trifft das nur in einem kleinen Bereich zu. Für die Neutralgasreibung $F\ix{N}$ gilt, dass diese bei großen thermischen Geschwindigkeiten des Staubes nahezu konstant wird, jedoch für "`kalten"' Staub mit zunehmendem Drift $v\ix{S}$ an Einfluss gewinnt. Die Ionenreibung $F\ix{ion}$ ist unter diesen Umständen vergleichsweise klein.\\
    In \autoref{img:kräfterichtungen} sind schematisch die Kräfte aus \autoref{img:kräfte} mit ihren Orientierungen dargestellt worden. Abbildungsteil \fett{(a)} zeigt dabei den Einfang von $\unit{\upmu m}$-großen Teilchen in der kleinen, lokalisierten Randschicht, wie es im diesem Experiment der Fall ist. Diese ordnen sich dabei in ausgedehnten Schichten mit hexagonaler Struktur (\tilt{fcc} bzw. \tilt{bcc} für ausgedehnte 3D-Cluster) an. Die Kräfte der Thermophorese und des "`Ionenwindes"' ($F\ix{ion}$) zeigen dabei aus dem Plasma heraus in Richtung der Kammer bzw. der Elektroden. Die Ionen sind dabei bestrebt, aufgrund der ambipolaren Diffusion in der Randschicht auf die Kammerwand, den Ladungsunterschied (etwa innerhalb einer \tilt{Debye}-Länge) zwischen dem Plasma und dem Gehäuse auszugleichen. Das liegt u.a. an den unterschiedlichen Beweglichkeiten der Ladungsträger und deren Strömung auf die Kammer (siehe Abschnitt \ref{sub:rand}). Weiterhin gilt ähnliches für die thermophoretische Kraft: das aufgeheizte Plasma erzeugt einen Impulsübertrag in Richtung der kühleren Kammer, woraus eine Kraft vom Plasma weg folgt. Der Staub wird somit in einem Teil der Entladung eingefangen, in dem diese Kräfte eine untergeordnete Rolle spielen. Die relevante elektrische Feldkraft ist besonders in der Nähe der Elektroden groß und kann damit dort Gravitationskraft bzw. $F\ix{th}$ und $F\ix{ion}$ kompensieren. Sie zeigt zudem in Richtung des Plasmas, da dieses in der Randschicht nicht mehr als neutral betrachtet werden kann.\\
    Im zweiten Teil (b) von \autoref{img:kräfterichtungen} ist ein Plasma mit Staubteilchen im $\unit{nm}$-Bereich unter Schwerelosigkeit gezeigt. Dabei entsteht der sog. \tilt{void}, welcher ein Fremdteilchen-freier Bereich in Mitten der Entladung ist und aufgrund der relativen Orientierungen der Kräfte zustande kommt. Für detailliertere Ausführungen siehe \cite{Dorier95}, \cite{Morfill99} und \cite{Goree99a}.

      \begin{figure}[!t]
        \centering
        \begin{subfigure}[b]{0.45\textwidth}
          \includegraphics[width=0.9\textwidth,height=0.9\textwidth]{figs/directionsofforcesandtrappingmelzerlinks.png}
          \caption{}
          \label{img:linksdirection}
        \end{subfigure}
        \begin{subfigure}[b]{0.45\textwidth}
          \includegraphics[width=0.9\textwidth,height=0.9\textwidth]{figs/directionsofforcesandtrappingmelzerrechts.png}
          \caption{}
          \label{img:rechtsdirection}
        \end{subfigure}
        \caption{Schemata für dein Einfang von Staub mit \fett{(a)}: $a\approx\unit[\tenpo{-}]{m}$ (\fett{b)}: $a\approx\unit[\tenpo{-7}]{m}$ bzw. unter Mikrogravitation. Es wurden die wichtigsten Kräfte und deren Richtungen eingezeichnet (nach \cite{Melzer12}).}
        \label{img:kräfterichtungen}
      \end{figure}

\newpage

  \section{Finite Yukawa-Cluster}\label{sub:yukawaclust}

      \begin{figure}[!h]
        \centering
        \begin{subfigure}[b]{0.3\textwidth}
          \centering
          \includegraphics[width=\textwidth,height=\textwidth]{figs/rot00013ungestrt.png}
          \caption{}
          \label{img:rechts}
        \end{subfigure}
        \begin{subfigure}[b]{0.3\textwidth}
          \centering
          \includegraphics[width=\textwidth,height=\textwidth]{figs/gruen05006ungestrt.png}
          \caption{}
          \label{img:links}
        \end{subfigure}
        \begin{subfigure}[b]{0.3\textwidth}
          \centering
          \includegraphics[width=\textwidth,height=\textwidth]{figs/gelb00011ungestrt.png}
          \caption{}
          \label{img:oben}
        \end{subfigure}
        \caption{Aufnahmen aus 3 verschiedenen, orthogonalen Raumrichtungen eines \tilt{Yukawa-Balls} mit $N=26$ Teilchen. Das System befindet sich bei niedrigen Gasdrücken ($\approx\unit[6-10]{Pa}$) in einer Argon-Entladung unter einem Kupferring, welcher sich im Plasma auf $\Phi\ix{fl}$ aufgeladen hat. \fett{(a)}: Ansicht aus der Ebene des Clusters \fett{(b)}: von oben \fett{(c)}: andere Richtung in der Ebene}
      \end{figure}

    Bisher wurden in den Abschnitten \ref{sub:kaprfplasm} und \ref{sub:rand} allgemein gültige Charakteristika des für diesen Aufbau verwendeten Plasmas besprochen. Außerdem sind in \ref{sub:ströme} und \ref{sub:dynamik} die für komplexe bzw. staubige Plasmen spezifischen Kenngrößen und Prozesse, wie beispielsweise Aufladung und Einfang, beschrieben worden. Damit kennen wir die Dynamik eines einzelnen Staubteilchen in einem Plasma und unter welchen Bedingungen ein Einfang gegeben ist, jedoch wissen wir nichts über das kollektive Verhalten der bereits besprochenen \tilt{Yukawa-Cluster}. Wann und wo sind diese - falls ein solcher Zustand existiert - stabil? Wie ist ihr Verhalten bei Störungen des externen Potentials? Dieser Abschnitt soll sich mit diesen Fragen, in Bezug auf einen Versuch unter den Bedingugen aus \autoref{img:linksdirection}, auseinander setzen.

       \subsection{Struktur}

           \begin{figure}[!t]
            \centering
            \begin{subfigure}[t]{0.48\textwidth}
              \centering
              \includegraphics[width=1.1\textwidth,height=0.3\textheight]{figs/einfangpotnu.png}
              \caption{}
              \label{img:potential}
            \end{subfigure}
            \begin{subfigure}[t]{0.48\textwidth}
              \centering
              \includegraphics[width=0.9\textwidth,height=0.3\textheight]{figs/gammaphasetransmelzer.png}
              \caption{}
              \label{img:gamma}
            \end{subfigure}
            \caption{\fett{(a)}: Harmonischer Einfang des Potentials $V$ in $\unit{eVs}$ mit globalem Minimum (dunkel). Darstellung für eine Elektrodenhöhe (siehe später Abschnitt \ref{sub:einfang}) von $\unit[1,5]{cm}$. \fett{(b)}: Phasendiagramm nach \cite{Melzer12} für eine effektive Coulomb-Kopplung nach \autoref{eq:kopplung}.}
           \end{figure}

%							\begin{wrapfigure}{r}{0.5\textwidth}
%								\begin{minipage}{0.2\textwidth}
%									\caption{\underline{\fett{oben}:} $N=190$-Cluster; Teilchen-Verteilung. \underline{\fett{unten}:} Vierte Schale mit penta- und hexagonalen Zellen, sowie Defekten (weiß).}\label{img:strukturN190}
%								\end{minipage}
%								\begin{minipage}{0.23\textwidth}
%									\centering
%									\includegraphics[width=\textwidth,height=2.6\textwidth]{figs/yukawaballN190.png}
%								\end{minipage}
%							\end{wrapfigure}

         Ein \tilt{Yukawa-Ball} kann durch das Erzeugen eines externen Potentials eingefangen werden: beispielsweise durch räumliche Begrenzung mit einer Küvette oder einem Ring, welche sich im Plasma aufladen und damit eine nach innen gerichtete, elektrische Kraft auf den ebenfalls negativ geladenen Cluster ausüben. Das aus den Kräften auf die Staubpartikel resultierende Potential kann als harmonisch angenommen werden, e.g.:

           \begin{align}
            V\left(\vec{r}\ix{j}\right)=\frac{m\ix{S}\omega\ix{0}^2}{2}\begin{pmatrix} 1 \\ \alpha\ix{y}  \\ \alpha\ix{z} \end{pmatrix} \vec{r}\ix{j}^2 \,\, ,
           \end{align}

        wobei $\alpha\ix{k}=\omega\ix{k}^2/\omega\ix{x}^2$ die relative der Richtung $k\in\left\lbrace x,y,z\right\rbrace$ und $\omega\ix{0}=\omega\ix{x}$ die absolute Einfangstärke ist. In \autoref{img:potential} ist das Potential, in dessen globalen Minimums sich der Cluster in diesem Experiment bildet, für eine feste Höhe über der getriebenen Elektrode dargestellt. Für eine geeignete Anordnung strebt man einen isotropen Einfang mit $\omega\ix{x}=\omega\ix{y}=\omega\ix{z}$ an. Ein finiter Yukawa-Cluster hat die Struktur konzentrischer Kugelschalen, welche mit steigendem Radius auch mit immer mehr Teilchen besetzt sind. Ein Visualisierung ist für $N=190$ in \autoref{img:strukturN190} gezeigt.

          \begin{align}
            \Gamma=\frac{Z\ix{S}e^2}{4\pi\varepsilon\ix{0}b\ix{WS}k\ix{B}T\ix{S}}  \,\,; \quad \Gamma\ix{C,eff}=\Gamma\exp\left(-\frac{b\ix{WS}}{\lambda\ix{D}}\right) \label{eq:kopplung}
          \end{align}

          \begin{wrapfigure}{l}{0.3\textwidth}
            \centering
            \includegraphics[width=0.22\textwidth,height=0.55\textwidth]{figs/yukawaballN190.png}
            \caption{Teilchen-Verteilung und Schalenstruktur, $N=190$ (nach \cite{Arp04})}\label{img:strukturN190}
            \vspace{-0.5cm}
          \end{wrapfigure}

         Um ein Maß für die Stabilität eines staubigen Plasmas zu erhalten, geht man, wie bei Festkörpern und deren Elektronengasen, von einer Punktladung $Q$ vor dem Hintergrund der Ionen und Elektronen aus (\tilt{one component plasma} - OCP). Der Kopplungsparameter $\Gamma$ in \autoref{eq:kopplung} beschreibt somit die elektrostatische Wechselwirkung eines Teilchens mit seinen Nachbarn in Einheiten der thermischen Energie.\\
        Für ein $\Gamma>1$ spricht man von einer starken Kopplung innerhalb des Clusters bzw. des Plasmas. Mit $\Gamma\geq\Gamma\ix{k}$ liegen kristalline Systeme vor. Bei einem kleineren Wert gehen die Cluster in einen flüssigen bzw. gasförmigen Zustand über (\autoref{img:gamma}). Das heißt, dass ein System aus Staubpartikeln "`schmelzen"' kann, bringt man durch Lasereinstrahlung o.ä. gezielt Energie in den Cluster ein und verringert damit die Ordnung bzw. erhöht die thermische Bewegung. Dabei verschwindet zuerst die Winkelabhängigkeit, was das Auflösen der kristallartigen Strukturen innerhalb des Clusters zur Folge hat (zum Beispiel \cite{Thomas96}).\\
        Die größe $b\ix{WS}=\left(3/4\pi n\ix{S}\right)^{-1/3}$ ist der \tilt{Wigner-Seitz}-Radius: er ist analog zu $\overline{b}$ aus Tab. \ref{tab:kenngroessen} eine Skala für Teilchenabstände. Insbesondere ist $b\ix{WS}$ eine korrektere Größe, da sich der Staub in hexagonalen bzw. pentagonalen Zellen auf den konzentrischen Sphären des Clusters anordnet. Die Zusammensetzung eines solchen \tilt{finiten Yukawa-Systems} ist in \autoref{img:strukturN190} dargestellt. Außerdem: sowohl aufgrund der bisher besprochenen Kräfte und der Coulomb-Wechselwirkung des Staubes, als auch wegen des nicht-reziproken \tilt{Ionenfokus} (siehe \cite{Melzer95c}) streben die Systeme bei entsprechenden Umgebungsparametern eine energetisch günstige, kristallartige Struktur an.  Der effektive Parameter für Coulomb-Wechselwirkungen $\Gamma\ix{C,eff}$ entspricht einer modifizierten Kopplung mit Rücksicht auf die Abschirmung durch die Ionenwolke (\cite{Lampe00}, \cite{Schweigert00d}) um ein Partikel bzw. den Cluster. Aus diesem Grund führt man die Abschirmstärke $\kappa=b\ix{WS}/\lambda\ix{D}$ ein, welche angibt, um wie viel die elektrostatische Wechselwirkung mit einem Teilchen der Ladung $Q$ innerhalb einer Elementarzelle der Staubpartikel abgeschwächt ist. Außerdem folgt daraus der Zusammenhang für die Phasengrenze in \autoref{img:gamma} $\Gamma\left(\kappa\right)$: für große \tilt{Wigner-Seitz}-Radien bzw. sehr kleine Debye-Längen verschwindet die Wechselwirkung zwischen den Staubteilchen nahezu vollständig. Das Potential geht für diesen Fall in das aus dem Modell harter Kugeln über (bspw. van-der-Waals-Gastheorie etc.).\\
        Abschließend sei erwähnt, das die bisher genannten Eigenschaften u.U. stark von der Art der Wechselwirkung bzw. der Teilchenzahl abhängen: die Betrachtungen gehen von reiner Coulomb- oder Yukawa-Wechselwirkung aus. Die Besetzungszahlen der Sphären des Clusters können jedoch für die verschiedenen Potentiale stark variieren, was u.a. eine Folge unterschiedlicher Abschirmungen ist.

      \subsection{Dynamik- und Modenanalyse} \label{subsub:moden}

        Auf Grundlage der Überlegungen zur Wechselwirkung und des Einfangs eines finiten Yukawa-Systems, kann nun eine Analyse der dynamischen Eigenschaften dessen erfolgen. Diese beruht auf der Entwicklung eines \tilt{Modenspektrums} dieses Clusters, wobei im Gegensatz zu ausgedehnten System darin nur eine endliche Zahl von Moden vorhanden sind. Statt einer Dispersionsrelation für Wellen bestimmt man demnach Schwingungsmoden, welche den Grenzen und Randbedingungen des Clusters bzw. des Einfangs genügen.\\
        Mit der Normierung auf Abstandseinheiten $r\ix{0}$ und Energieeinheiten $E\ix{0}$ folgt \autoref{eq:energie}. Die erste Summe stellt den kinetischen Anteil der Energie des $i$-ten Teilchens im Rahmen eines harmonischen Oszillators dar. Die Yukawa-Abstoßung im zweiten Teil kommt aus der elektrostatischen Wechselwirkung innerhalb des Staubkristalls, weswegen $\kappa=r\ix{0}/\lambda\ix{D}$ und $r\ix{ij}=|\vec{r}\ix{i}-\vec{r}\ix{j}|$ gilt.

          \begin{align}
            E=\sum_{i=1}^{N}\vec{r}\ix{i}\,^2+&\sum_{i<j}^{N}\frac{\exp\left(-\kappa r\ix{ij}\right)}{r\ix{ij}} \label{eq:energie} \\
            r\ix{0}=\left(\frac{2Z\ix{S}^2e^2}{4\pi\varepsilon\ix{0}m\ix{S}\omega\ix{0}^2}\right)^{1/3} \quad &\quad E\ix{0}=\left(\frac{Z\ix{S}^2e^4m\ix{S}\omega\ix{0}^2}{8\pi\varepsilon\ix{0}}\right)^{1/3} \nonumber
          \end{align}

        Ausgehend von der normierten Gesamtenergie $E$ eines N-Teilchen-Clusters lässt sich, als Analogon zur Taylor-Entwicklung, durch Approximation um ein Equilibrium die sog. \tilt{Hesse-Matrix} in \autoref{eq:matrix} als dynamische Matrix $A\in\text{Mat}\left(3N\times3N\right)$ aufstellen. Hinter dieser Formulierung steht die Idee, dass jede Bewegung als anteilige Überlagerung der Schwingungsmoden verstanden werden kann. Somit steht in $A$ für alle Teilchenkoordinaten $i,j$ die 2. Ordnung der Entwicklung um das Gleichgewicht. Löst man für diese das Eigenwertproblem (\cite{Schweigert95c}), so erhält man mit $\vec{\nu}\ix{p}$ und $\omega\ix{p}$ den Modenvektor und -frequenz der Mode $p$.

          \begin{align}
            A=
            \begin{pmatrix}
            \frac{\partial^2 E}{\partial x\ix{i} \partial x\ix{j}} & \cdots & \frac{\partial^2 E}{\partial x\ix{i} \partial z\ix{j}} \\ 
            \vdots & \ddots & \vdots \\ 
            \frac{\partial^2 E}{\partial z\ix{i} \partial x\ix{j}} & \cdots & \frac{\partial^2 E}{\partial z\ix{i} \partial z\ix{j}}
            \end{pmatrix} 
            \quad \text{mit} \quad \frac{\partial^2 E}{\partial x\ix{i}\partial y\ix{j}}&=
            \begin{pmatrix}
            \frac{\partial^2 E}{\partial x\ix{1} \partial y\ix{1}} & \cdots & \frac{\partial^2 E}{\partial x\ix{1} \partial y\ix{N}} \\ 
            \vdots & \ddots & \vdots \\ 
            \frac{\partial^2 E}{\partial x\ix{N} \partial y\ix{1}} & \cdots & \frac{\partial^2 E}{\partial x\ix{N} \partial y\ix{N}}
            \end{pmatrix}
            \label{eq:matrix} \\
            \left(A-\omega\ix{p}^2I_{(3N)}\right)\vec{\nu}\ix{p}&=0 \label{eq:ewp} \\ 
            \text{wobei} \quad \vec{\nu}\ix{p}^\top=\left(x\ix{1,p},\dots,x\ix{N,p},y\ix{1,p},\dots\right) \quad &\text{bzw.} \quad \vec{\nu}\ix{i,p}^\top=\left(x\ix{i,p},y\ix{i,p},z\ix{i,p}\right) \nonumber
          \end{align}

        Aus der Lösung von \autoref{eq:ewp} um $A$ erhält man folglich $3N$ Modenfrequenzen und -vektoren. Den Anteil, den eine jede Mode $p$ an der Bewegung eines Teilchens $i$ hat, erhält man durch die Projektion des Vektors der Geschwindigkeit $\vec{v}\ix{i}\left(t\right)$ auf die Modenvektoren $\vec{\nu}\ix{i,p}$. Anders herum: den Anteil der Mode an der Clusterbewegung, ob thermisch oder extern angeregt, erhält man aus der Summe aller Anteile an den Teilchenbewegungen (siehe \autoref{eq:summe}, \cite{Melzer03}). Die Energie, welche in der Mode $p$ bei der Frequenz $\omega$ gespeichert ist, ergibt sich über die Fouriertransformation dieses Bewegungsanteils zu $S\ix{p}\left(\omega\right)$.

          \begin{align}
            f\ix{p}&\left(t\right)=\sum_{i=1}^{N}\vec{v}\ix{i}\left(t\right)\vec{\nu}\ix{i,p} \label{eq:summe} \\
            S\ix{p}\left(\omega\right)&=\left.\left.\frac{2}{T}\right|\int_{0}^{T}f\ix{p}\left(t\right)\exp\left(-\imag\omega t\right)\diff t\right|^2 \label{eq:energiedicht}
          \end{align}

          \begin{figure}
            \centering
            \begin{subfigure}[b]{0.48\textwidth}
              \centering
              \includegraphics[width=\textwidth,height=0.3\textheight]{figs/modenvenergie3teilchenmelzer.png}
              \caption{}
              \label{img:spektrum}
            \end{subfigure}
            \begin{subfigure}[b]{0.48\textwidth}
              \centering
              \includegraphics[width=\textwidth,height=0.3\textheight]{figs/modenvektoren3teilchenmelzer.png}
              \caption{}
              \label{img:moden}
            \end{subfigure}
            \caption{\fett{(a)}: Modenspektrum für einen 3-Teilchen-Cluster. Die weißen Punkte entsprechen theoretisch errechneten Werten der Modenfrequenzen. Die Bereiche von $S\ix{p}\left(\omega\right)\rightarrow0$ treten bei der Eigenfrequenz auf. \fett{(b)}: Eigenvektoren ($\vec{\nu}\ix{i,p}$) für $N=3$. Die Modenfrequenz ist auf die Einfangstärke $\omega\ix{0}$ normiert (nach \cite{Melzer12}).}
          \end{figure}

        Die spektrale Energiedichte aus \autoref{eq:energiedicht} wird in \autoref{img:spektrum} für einen sehr simplen Cluster aus $N=3$ Teilchen dargestellt. Passend dazu sind in \autoref{img:moden} die Modenvektoren der drei Staubpartikel eingezeichnet. Insbesondere sind dort Moden gezeigt, welche allgemein für alle Cluster auftreten: \tilt{breathing}, \tilt{slosh} und Rotation. $p=1$ zeigt die \tilt{breathing}-Mode: alle Teilchen entfernen sich vom Clusterschwerpunkt, wobei dieser jedoch fest bleibt. Die Frequenz $\omega\ix{breath}$ hängt schwach von $N$ ab und steigt mit der Abschirmung $\kappa$. Mode $p=2$ ist die Rotationsmode. Ihre Eigenfrequenz ist $0$, da es für diese keine rückstellenden Kräfte gibt. In $p=3$ und $p=4$ sind \tilt{kink}-Moden gezeigt. Nummer 5 und 6 stellen \tilt{slosh}-Moden dar: das System führt eine Schwerpunktstranslation aus, wobei alle Partikel sich in die selbe Richtung bewegen.\\
        Diese \tilt{Normalmodenanalyse} ist ein geeignetes Mitte zur Analyse der Dynamik eines finiten Systems (siehe \cite{Melzer01},\cite{Melzer03}), vorausgesetzt es genügt den gemachten Voraussetzungen: Yukawa-Wechselwirkung, starke Kopplung, homogene Partikel \dots

      \subsection*{Fluidmoden}

        Genau wie in der, diesem Teilgebiet übergeordneten Plasmaphysik, lässt sich ein finiter Cluster als hydrodynamische Gesamtheit auffassen. Ähnlich wie beim \tilt{Fluidmodell} und der Betrachtung durch die \tilt{magnetohydrodynamischen Gleichungen} (MHD), nimmt man hierbei das System als kontinuierliche Masse mit entsprechender Ladungsverteilung an. Daraus ergeben sich neue Möglichkeiten die Dynamik des Cluster zu beschreiben und nachzuvollziehen.\\
        Eingangs muss das neue Potential $\Phi\left(\vec{r},t\right)$ des fluiden "`Tropfens"' beschrieben werden. Da der Cluster von Ladungsträgern abgeschirmt wird, welche sich um diesen auf Grund seiner elektrostatischen Wechselwirkung ansammeln - analog zu den vorherigen Abschnitten -, kann man die modifizierte Poissongleichung (\autoref{eq:poisson}) für die Ladungsdichte $\rho\left(\vec{r},t\right)$ des Systems benutzen. Deren Lösung erhält man mit der \tilt{Green-Funktion} $G\left(\vec{r}\right)$ aus \autoref{eq:green} in (\ref{eq:fluid}). 

          \begin{align}
            &\left(\Delta-\kappa^2\right)\Phi\left(\vec{r},t\right)=\frac{1}{\varepsilon\ix{0}}\rho\left(\vec{r},t\right) \label{eq:poisson} \\
            \Phi\left(\vec{r}\right)&=\int_{\mathbb{R}^3}\frac{\rho\left(\vec{r}\,^\prime,t\right)}{4\pi\varepsilon\ix{0}|\vec{r}-\vec{r}\,^\prime|}\diff^{3}r^\prime \\
            &\,\,\,\,\vdots\quad G\left(\vec{r}\right)=\frac{\exp\left(-\kappa|\vec{r}|\right)}{4\pi|\vec{r}|} \label{eq:green}\\
            \Phi\left(\vec{r}\right)&=\int_{\mathbb{R}^3}\frac{\exp\left(-\kappa|\vec{r}-\vec{r}\,^\prime|\right)}{4\pi\varepsilon\ix{0}|\vec{r}-\vec{r}\,^\prime|}\rho\left(\vec{r}\,^\prime,t\right)\diff^{3}r^\prime \label{eq:fluid}
          \end{align}

        Die Green-Funktion entspricht einem einheitenlosen Yukawa-Potential, welches man nach \cite{Arfken05},\cite{Yap10} in eine Reihe von sphärisch-harmonischen \tilt{Kugelflächenfunktionen} $Y_{lm}\left(\theta,\varphi\right)$ und \tilt{modifizierten Besselfunktionen} $i_{l}\left(x\right)$ und $k_{l}\left(x\right)$ entwickeln kann. Die Definitionen erfolgen in dreidimensionalen Kugelkoordinaten $\left(r,\theta,\varphi\right)$, wobei die minimalen bzw. maximalen Abstände $r\ix{<}=\min\left(r^\prime,r\right)$ und $r\ix{>}=\max\left(r^\prime,r\right)$ sind.

          \begin{align}
            \frac{\exp\left(-\kappa|\vec{r}-\vec{r}\,^\prime|\right)}{4\pi|\vec{r}-\vec{r}\,^\prime|}=\sum_{l=0}^{\infty}\sum_{m=-l}^{l}\frac{r\ix{<}^l}{r\ix{>}^{l+1}}i_{l}\left(\kappa r\ix{<}\right)k_{l}\left(\kappa r\ix{>}\right)\exp\left(-\kappa r\ix{>}\right)Y_{lm}^*\left(\theta^\prime,\varphi\prime\right)Y_{lm}\left(\theta,\varphi\right) \label{eq:entwick}
          \end{align}

        Nach \cite{Ivanov09} und \cite{Schella13} lassen sich aus \autoref{eq:entwick} die Multipolmomente der Ladungsdichte des Clusters gewinnen (siehe \autoref{eq:multipol}). Die Indizes $lm$ geben die Modenzahlen an, wobei analog zu den Zust\"anden eines Wasserstoff-artigen Atoms $\Psi\left(\vec{r}\right)=L_{n}\left(r\right)Y_{lm}\left(\theta,\varphi\right)$ diese die Symmetriezahlen bzw. -richtungen der Schwingungen des "`Tropfens"' angeben (siehe \autoref{img:fluidmode}).

          \begin{align}
            \rho_{lm}\left(t\right)=&\sqrt{\frac{4\pi}{2l+1}}\int_{\mathbb{R}^{3}}\rho\left(\vec{r},t\right)i_{l}\left(\kappa r\right)Y^{*}_{lm}\left(\theta,\varphi\right)\diff^{3}r \label{eq:multipol} \\
            Q_{lm}\left(\omega\right)=&\left.\left.\frac{2}{TN}\right|\int_{0}^{T}\rho_{lm}\left(t\right)\exp\left(-\imag\omega t\right)\diff t\right|^{2}
          \end{align}

        Wie in \autoref{eq:energiedicht} wird \"uber die Fouriertransformation aus dem Zeit- in den Frequenzraum die anteilige Energie der Mode $q_{lm}$ an der Schwingungsbewegung des Tropfens bei $\omega$ in der korrespondierenden Periode  $T$ berechnet (\cite{Schella13}). Das Frequenzpektrum stellt somit die Zusammensetzung der Clusterbewegung aus den Moden $lm$ dar.\\
        Nach \autoref{eq:multipol} k\"onnen die einzelnen Moden als Dipol, Quadrupol, \dots identifiziert werden. Ihren Symmetriezahlen $lm$ entsprechend, vollf\"uhrt der Cluster dabei die bereits bekannten Schwingungen (siehe \ref{subsub:moden}): die \tilt{breathing}- und \tilt{slosh}-Mode. Dabei ist das Tupel $\left(l=0,m=0\right)$ die Monopolschwingung: eine periodischen Kompression und Relaxation des Systems (\tilt{breathing}). Die Indizes $\left(l=0,m=1\right)$ und deren Permutation stellen eine Dipolschwingung dar. Der Ball vollf\"uhrt eine Schwerpunktstranslation (\tilt{slosh}) und leichte Deformation, senkrecht zur Schwingungsachse. Die Beschreibung einer Rotationsbewegung fehlt, da die Ladungsverteilung invariant gegen\"uber Drehungen ist. H\"ohere Symmetrien - Schwingungen auf 2 oder mehr Raumachsen - entsprechen demnach h\"oheren Werten von $\left(l,m\right)$. Die in diesem Versuch betrachteten Bewegungen sind Quadrupol- bzw. Dipolschwingungen.

          \begin{figure}[H]
                          \centering
                          \includegraphics[width=\textwidth,height=0.375\textwidth]{figs/y001122.png}
                          \caption{Darstellung der Fluidmoden über Kugelflächenfunktionen mit Pfeilen als charakteristische Schwingungsrichtungen bzw. -achsen. Rote Bereiche sind als Regionen hoher Dichten, blaue als jene niedrigerer zu interpretieren. Von links nach rechts: Monopol (\tilt{breathing}), Dipol (\tilt{slosh}), Quadrupol (nach \cite{Mulsow13}).}
                          \label{img:fluidmode}
                      \end{figure}

